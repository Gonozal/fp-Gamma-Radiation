% Klassifiziert den Dokumenten-Typ
% Doku: http://exp1.fkp.physik.tu-darmstadt.de/tuddesign/
% Farben: http://www.tu-darmstadt.de/media/medien_stabsstelle_km/services/medien_cd/das_bild_der_tu_darmstadt.pdf
%  bigchapter: Chapter haben doppelte Schriftgröße
%  linedtoc: Linien im Inhaltsverzeichnis wie bei Überschriften
%  colorbacktitle: Der Dokumenten-Titel wird mir der Accentfarbe hinterlegt
\documentclass[bigchapter,colorback,accentcolor=tud4b,linedtoc,11pt]{tudreport}

% Input Dokument hat das Encoding UTF-8
\usepackage[utf8]{inputenc}
% Wichtiges Paket für Links und verlinktes Inhaltsverzeichnis
\usepackage[ngerman]{hyperref}
% Paket für Fußnoten
\usepackage[stable]{footmisc}
\usepackage{multirow}
\usepackage{colortbl}
% Alternatives package für bilder
\usepackage{wrapfig}
% Paket für amsmath (aligned mathe formeln)
\usepackage{amsmath}
% Farbige tabellen
\usepackage{colortbl}
 
% Paket für Bibliotheks-Verzeichnis, square: Verwende eckige statt runde klammern
% \usepackage[square]{natbib}
% Paket zum Plotten von Datensätzen
\usepackage{pgfplots}
\pgfkeys{%
  /pgfplots/default/.style={%
    /pgf/number format/use comma,
    legend pos=north east,
    x tick label style={/pgf/number format/1000 sep=},
    y tick label style={/pgf/number format/1000 sep=},
    width=0.9\linewidth,
    height=0.40\linewidth,
    scale only axis,
    grid=both,
    tick align=outside,
    tickpos=minor,
    left X tick num=3,
    minor y tick num=4,
    minor grid style={dotted,thin}
  }
}

% Anhänge für Original-Messdaten
\usepackage{fancyvrb}

% Verwende deutsche Bezeichner für Inhaltsverzeichnis, ... (ngerman = New German: neue Rechtschreibung)
\usepackage{ngerman}
% Deutsche Zahlen (entfernt z.B. das Leerzeichen nach einem Dezimal-Komma)
\usepackage{ziffer} 

\usepackage[verbose]{placeins}

%wegen Grafikverschiebung hinzugefügt
\usepackage{float}

%\usepackage{graphicx}
%\usepackage{caption}
\usepackage{subcaption} %Für subfigures

% PDF-Optionen
\hypersetup{%
  pdftitle={TU Darmstadt \- Physikalisches Praktikum für Fortgeschrittene},
  pdfauthor={Esra Bauer, Sören Link},
  pdfsubject={Versuch 2.2-A},
  pdfview=FitH,
}
% Nummeriere formeln in Subsections einzeln
% Kleines makro zur assymetrischen Fehlerangabe

% Entspricht-Zeichen
\usepackage{scalerel}

\newcommand\equalhat{%
\let\savearraystretch\arraystretch
\renewcommand\arraystretch{0.3}
\begin{array}{c}
\stretchto{
    \scalerel*[\widthof{=}]{\wedge}
    {\rule{1ex}{3ex}}%
}{0.5ex}\\ 
=%
\end{array}
\let\arraystretch\savearraystretch
}
%BEGINN TITELSEITE

\title{Interaction of $\gamma$-Radiation with Matter}

\subtitle{Esra Bauer  \\Sören Link}

\subsubtitle{Tutor: Haridas Pai \hfill 4/27/2015}

\author{Esra Bauer, Sören Link, Christian Hoch}

%\settitlepicture{img/title.jpg}

\institution{Physikalisches Praktikum \\für Fortgeschrittene \\ Versuch 2.2-A}

\date{\today}
%ENDE TITELSEITE


\begin{document}
%ANFANG DOKUMENT

%Titelseite einfügen
\maketitle

%Inhaltsverzeichnis einfügen
\tableofcontents

%ANFANG INHALT
\chapter{Introduction}



\chapter{Theoretical Background}


\section{Radioactice Decay}

\section{Interaction of $\gamma$-rays with Matter}

\section{Scintillation Spectroscopy}


\chapter{Experimental Setup and Execution}

\section{Setup}

\section{Execution}

\chapter{Evaluation}
\section{Analysis of spectrum of $^{137}Cs$}
\subsection{Energy-calibration}
\subsection{Energy spectrum}
\subsection{Backscattering peak}
\subsection{Abosrption by a lead plate}
\subsection{Convection coefficient}

\section{Analysis of spectrum of $^{22}Na$}
\subsection{Energy-calibration}
\subsection{Energy spectrum}
\subsection{Probability of sumpeaks}
\subsection{$\beta^+$-branching ratio}
\subsection{Multiple event process}

\section{Energy Resolution}

\section{Background Radiation}

\chapter{Conclusion}

%ENDE INHALT
\cleardoublepage{}
% Eintrag fürs Inhaltsverzeichnis
\newpage
\begin{thebibliography}{100}
  \bibitem{Anleitung} {Experimental Instructions}
  \bibitem{na22decay} {Semibyte, homepage of physics lab assistent and qualified
      computer scientist Tobias Krähling:
      \url{http://www.semibyte.de/wp/download/graphicslib/physics/termschema_na22.png}
    [CC BY-NC-SA 3.0]}
  \bibitem{cs137decay} {Wikipedia, the free encyclopedia By Tubas-en [Public
      domain], via Wikimedia Commons: \url{http://commons.wikimedia.org/wiki/File%3ACs-137-decay.svg}}
\end{thebibliography}
\end{document}

%%% Local Variables:
%%% mode: latex
%%% TeX-master: t
%%% End:
