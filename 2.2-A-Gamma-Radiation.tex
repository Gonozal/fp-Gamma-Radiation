% Klassifiziert den Dokumenten-Typ
% Doku: http://exp1.fkp.physik.tu-darmstadt.de/tuddesign/
% Farben: http://www.tu-darmstadt.de/media/medien_stabsstelle_km/services/medien_cd/das_bild_der_tu_darmstadt.pdf
%  bigchapter: Chapter haben doppelte Schriftgröße
%  linedtoc: Linien im Inhaltsverzeichnis wie bei Überschriften
%  colorbacktitle: Der Dokumenten-Titel wird mir der Accentfarbe hinterlegt
\documentclass[bigchapter,colorback,accentcolor=tud4b,linedtoc,11pt]{tudreport}

% Input Dokument hat das Encoding UTF-8
\usepackage[utf8]{inputenc}
% Wichtiges Paket für Links und verlinktes Inhaltsverzeichnis
\usepackage[ngerman]{hyperref}
% Paket für Fußnoten
\usepackage[stable]{footmisc}
\usepackage{multirow}
\usepackage{colortbl}
% Alternatives package für bilder
\usepackage{wrapfig}
% Paket für amsmath (aligned mathe formeln)
\usepackage{amsmath}
% Farbige tabellen
\usepackage{colortbl}
 
% Paket für Bibliotheks-Verzeichnis, square: Verwende eckige statt runde klammern
% \usepackage[square]{natbib}
% Paket zum Plotten von Datensätzen
\usepackage{pgfplots}
\pgfkeys{%
  /pgfplots/default/.style={%
    /pgf/number format/use comma,
    legend pos=north east,
    x tick label style={/pgf/number format/1000 sep=},
    y tick label style={/pgf/number format/1000 sep=},
    width=0.9\linewidth,
    height=0.40\linewidth,
    scale only axis,
    grid=both,
    tick align=outside,
    tickpos=minor,
    left X tick num=3,
    minor y tick num=4,
    minor grid style={dotted,thin}
  }
}

% Anhänge für Original-Messdaten
\usepackage{fancyvrb}

% Verwende deutsche Bezeichner für Inhaltsverzeichnis, ... (ngerman = New German: neue Rechtschreibung)
\usepackage{ngerman}
% Deutsche Zahlen (entfernt z.B. das Leerzeichen nach einem Dezimal-Komma)
\usepackage{ziffer} 

\usepackage[verbose]{placeins}

%wegen Grafikverschiebung hinzugefügt
\usepackage{float}

%\usepackage{graphicx}
%\usepackage{caption}
\usepackage{subcaption} %Für subfigures

% PDF-Optionen
\hypersetup{%
  pdftitle={TU Darmstadt \- Physikalisches Praktikum für Fortgeschrittene},
  pdfauthor={Esra Bauer, Sören Link},
  pdfsubject={Versuch 2.2-A},
  pdfview=FitH,
}
% Nummeriere formeln in Subsections einzeln
% Kleines makro zur assymetrischen Fehlerangabe

% Entspricht-Zeichen
\usepackage{scalerel}

\newcommand\equalhat{%
\let\savearraystretch\arraystretch
\renewcommand\arraystretch{0.3}
\begin{array}{c}
\stretchto{
    \scalerel*[\widthof{=}]{\wedge}
    {\rule{1ex}{3ex}}%
}{0.5ex}\\ 
=%
\end{array}
\let\arraystretch\savearraystretch
}
%BEGINN TITELSEITE

\title{Interaction of $\gamma$-Radiation with Matter}

\subtitle{Esra Bauer  \\Sören Link}

\subsubtitle{Tutor: Haridas Pai \hfill 4/27/2015}

\author{Esra Bauer, Sören Link, Christian Hoch}

%\settitlepicture{img/title.jpg}

\institution{Physikalisches Praktikum \\für Fortgeschrittene \\ Versuch 2.2-A}

\date{\today}
%ENDE TITELSEITE


\begin{document}
%ANFANG DOKUMENT

%Titelseite einfügen
\maketitle

%Inhaltsverzeichnis einfügen
\tableofcontents

%ANFANG INHALT
\chapter{Introduction}
In this experiment we analyze the radioactive Isotopes $^{22}Na$ and
$^{137}Cs$. This includes determining the convection coefficient $\alpha_K$ of
$^{137}Cs$ and the $\beta^+$ branching coefficient of $^{22}Na$.

\chapter{Theoretical Background}
\section{Radioactice Decay}

Additionally to the well known $\alpha$, $\beta^{\pm}$ and $\gamma$ decay there are two other radioactive decay types which are important for us. First we have electron capture which gives the same progeny as $\beta^+$ decay. The nucleus absorbs an inner atomic electron whereby a proton is converted into a neutron and an electron neutrino. Released energy is given to the neutrino as kinetic energy resp. the progeny stays in an excited state if energy is left over. Because the missing electron is replaced by other atomic electrons there is spontaneous X-ray emission.

Second there is the inner conversion as an alternative to $\gamma$ decay. Via direct elektromagnetic interaction energy is transferred from the nucleus to an atomic electron. The nucleus switches into an lower state and the electron is emitted i.e. the atom is ionized. As a characteristic number we define the conversion rate as follows

$$\alpha = \frac{N_{e^-}}{N_{\gamma}}$$

where $N_{e^-}$ is the number of emitted conversion electrons and $N_{\gamma}$ is the number of emitted photons.

Below we see the decay schemes of the used Isotopes:

\begin{figure}[H]
    \centering
    \begin{subfigure}[H]{0.44\textwidth}
        \includegraphics[width=\textwidth]{img/termschema_na22.png}
        \caption{Decay scheme of $^{22}Na$ \cite{na22decay}.}
        \label{fig:gull}
    \end{subfigure}%
    \qquad
    ~ %add desired spacing between images, e. g. ~, \quad, \qquad, \hfill etc.
        %(or a blank line to force the subfigure onto a new line)
    \begin{subfigure}[H]{0.44\textwidth}
        \includegraphics[width=\textwidth]{img/Cs-137-decay.png}
        \caption{Decay scheme of $^{22}Na$ \cite{cs137decay}.}
        \label{fig:tiger}
    \end{subfigure}
    \caption{Decay scheme of the radiactive isotopes used in this experiment.}
\end{figure}

\section{Interaction of $\gamma$-rays with Matter}
The interaction of $\gamma$-rays with matter consists of a superposition
of three different effects. At low photon energies their interaction with matter
is dominated by the photoelectric effect where photons are completely absorbed
by electrons. Bound electrons that absorb a photon in this way are broken free
from their atom, creating an electric current.

Photons with energies in orders of magnitudes ranging from $10^2 keV$ to $10^3 keV$
mostly interact through elastic scattering with weakly bound electrons. This
results in a transfer of energy from the photon to the electron and is called
the Comtpon effect.

Above energies of $1022keV$ photons traveling in a strong coloumb field can
spontanously convert into an electron and a positron. In doing so the photon
ceases to exists and converts it's energy first into the mass of the $e^-$ and
$e^+$ particle. The remaining energy is then being transferred onto these
particles in the form of kinetic energy. Because the impulse has to be conserved
during this conversion, the effect most often occurs near the nucleus, which can
absorb the impulse of the incident photon. This effect is called pair production
and the likelyhood of it occuring increases with higher photon energies.



\section{Scintillation Spectroscopy}


\chapter{Experimental Setup and Execution}
\begin{figure}[H] 
  \centering
     \includegraphics[width=0.8\textwidth]{img/aufbau.png}
     \caption{Experimental setup. As source material we used $^{22}Na$ and
       $^{137}Cs$. In addition we mounted plates of aluminium, cadmium, iron and
     lead behind the source to measure the backscatter radiation. Finally we have
     also measured the background radiation without a radiactive source \cite{Anleitung}.}
  \label{fig:aufbau}
\end{figure}

The experimental setup remains mostly constant throughout the experiment. First
we measured the radioactive spectrum of $^{137}Cs$ with a distance $h$ from the
detecter of $10 cm$. During these measurements
we mounted plates of aluminium, cadmium, iron and lead behind the source to
determine their effect on the spectrum. We've also measured the spectrum of
$^{137}Cs$ through a lead shield.

Afterwards we replaced the caesium isotope with radioactive $^{22}Na$ and
measured it's spectrum twice. The first measurement was done with the natrium
sample mounted $10 cm$ above the scintillation detector. For the second
measurment we put the sample directly on top of the detector.

Each measurement concluded when the main peak of the spectrum reached $3000$ counts.

In the end we measured the background radiation for 31 minutes and 40 seconds.

\chapter{Evaluation}
\section{Analysis of spectrum of $^{137}Cs$}
\subsection{Energy-calibration}
\subsection{Energy spectrum}
\subsection{Backscattering peak}
\subsection{Abosrption by a lead plate}
\subsection{Convection coefficient}

\section{Analysis of spectrum of $^{22}Na$}
\subsection{Energy-calibration}
\subsection{Energy spectrum}
\subsection{Probability of sumpeaks}
\subsection{$\beta^+$-branching ratio}
\subsection{Multiple event process}

\section{Energy Resolution}

\section{Background Radiation}

\chapter{Conclusion}

%ENDE INHALT
\cleardoublepage{}
% Eintrag fürs Inhaltsverzeichnis
\newpage
\begin{thebibliography}{100}
  \bibitem{Anleitung} {Experimental Instructions}
  \bibitem{na22decay} {Semibyte, homepage of physics lab assistent and qualified
      computer scientist Tobias Krähling:
      \url{http://www.semibyte.de/wp/download/graphicslib/physics/termschema_na22.png}
    [CC BY-NC-SA 3.0]}
  \bibitem{cs137decay} {Wikipedia, the free encyclopedia. By Tubas-en [Public
      domain], via Wikimedia Commons: \url{http://upload.wikimedia.org/wikipedia/commons/thumb/3/3e/Cs-137-decay.svg/500px-Cs-137-decay.svg.png}}
\end{thebibliography}
\end{document}

%%% Local Variables:
%%% mode: latex
%%% TeX-master: t
%%% End:
